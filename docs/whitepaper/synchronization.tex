\chapter{Synchronization}
This chapter give a brief introduction of synchronization basics as well as how
OpenSync works and handles real life synchronization issues.\\
\\
Different synchronization techniques used nowadays, which have some of the
following tasks in common:
\begin{itemize}
 \item Connect
 \item Get changes
 \item Conflict resolution
 \item Multiply changes
 \item Commit changes
 \item Disconnect
\end{itemize}
Those tasks are in common for synchronization technique/protocol, but differ in
detail to fit the needs for different circumstances to meet the best efficiency.
Such circumstances could be:
\begin{itemize}
 \item Number of synchronization parties. If the number of synchronization
 parties is two, then multiplying of changes is just simple duplication of the
 change.
 \item Unidirectional/Bidirectional synchronization. On unidirectional
 synchronization no conflict resolution required for two parties.
 \item Resource. Depending of the type of data resource further work is required
 to get changes. Is the resource able to tell which data changed since the last
 synchronization, by its own? Or is further help/facility required to detect
 which data changed since the last sync. Example: file systems, databases,
 stacked data in a single file, ...
 \item Type of data. Is the data in a consistent format and supported by all
 parties? File synchronization. Is the data not consistent and contains unique
 information which doesn't allow to do a binary compare? Weak compare? Is 
 conversion to a common format for different parties required?
 \item Protocol. Does the protocol require to read only the latest changes or 
 all at once? Does the protocol support single commits or only all at once 
 (batch commit)?
 \item Transport. Are various transport layer involved? Does it require to
 connect and disconnect in a specified way? Limited bandwidth? Example: 
 Bluetooth, USB, ...
\end{itemize}
You see, there lots of different circumstances which makes it quite complicated
to meet all the needs of different ways to synchronize and synchronization 
protocols.\\
This is also only the tip of the iceberg, since it describes only the
synchronization role of the ">Server"<. 

\section{Synchronization Role}
The term ">Server"< is quite confusing,
especially in the combination of a synchronization protocol which uses a
transport protocol based on the ">Client"<- and ">Server<"-Model. Most famous
example is ">SyncML"<, which support among others the HTTP and OBEX protocol as
transport. You might know ">HTTP Server"< like the Apache Webserver and ">HTTP
Client"< like the Firefox Webbrowser. In SyncML you can have for example (same 
for other transports supported by SyncML):
\begin{itemize}
\item HTTP Server transport and act as Synchronization Server
\item HTTP Client transport and act as Synchronization Server
\item HTTP Server transport and act as Synchronization Client
\item HTTP Client transport and act as Synchronization Client
\end{itemize}
OpenSync doesn't care much about Transport Server/Client role, this is
up to the Plugins. There is only a little detail which OpenSync have to care
about plugin when they're acting as the transport Server role, which is about
that the plugin has to be initialized all the time so the client can connect.
More about this in the Plugin chapter.\\
\\
Unfortunately OpenSync supports in version 0.40 only the Synchronization role
Server. The passive role as Synchronization Client isn't yet implemented, but is
on the top of the project agenda. The reason for this is that the current
implementation of synchronization tasks/steps mentioned above are currently
fixed. As Synchronization Client the order and number of this synchronization 
steps/tasks would differ to the Server role. More about this issue you can find
in the Framework Chapter in section Synchronization Role.

\section{Slow Sync}
Various Synchronization protocols are using so called ">Slow Sync"<
synchronization technique. This consists of two types of synchronizations, the
already mentioned ">Slow Sync"< and a regular Synchronization (sometimes called
">Fast Sync<"). The difference between the Slow and the regular (Fast) Sync is 
that the regular one only transfers changes since the last synchronization. 
This means on a regular synchronization not all entries have to be transfered, 
converted. This makes the synchronization quite efficient and very fast. The
so called ">Slow Sync"< requests intentionally all entries, which makes the
synchronization a bit slower. Additionally the Synchronization Framework has to
interpret every single entry/change as newly added, since the Framework vanished
in advance the entire mappings and has to compare every single reported entry
from each party and find the fitting counterpart. This and the combination of
transferring all entries makes the synchronization compared to the regular
(Fast) synchronization very slow. The ">Slow Sync"< is only used in certain
cases to avoid data inconsistence and to keep all the data in sync. ">Slow
Sync"< got emitted in following circumstances:
\begin{itemize}
\item First/Initial Synchronization
\item Party got reseted (same as first sync)
\item Party got synchronized in meanwhile within another environment 
\item After an aborted/failed synchronization
\end{itemize}
\section{Object Types}
The term ">Object Types"< is in OpenSync used to describe the type/category of 
data. Example for ">Object Types"< are Contacts, Events, Todos, Notes or plain 
Data (like the content of a file) and others. (It's not limited to PIM Data!). 
Those Object Types get separated processed, to make it configurable which 
Object Type should get synchronized. Example: Only synchronize contacts of the 
mobile, no events, todos nor notes.
\section{Formats}
The ability to synchronize different Parties which use different formats, makes
the OpenSync Framework to a very powerful Synchronization Framework. In OpenSync
each Format is associated with one Object Type (see previous chapter). This
Object Type as common denominator for different formats makes it possible to
determine a conversion path between different formats. The conversion path
consists of various format converters, which are provided by Format Plugins.
Example: Two parties should synchronize their contacts (the Object Type). Party
A stores the contacts as VCard 3.0 and Party B stores the contacts in some
Binary Format. To synchronize the VCard 3.0 and the Random Binary Contact Format
format plugins have to register those formats and provide converters. The
Plugins don't have to provide converters for every known Format, often a certain
amount of converters to common formats or a common denominator format is enough
to create a conversion path between VCard 3.0 to Binary Contact Format.
\section{Mappings}
If an entry got changed on one Party, the logical same entry has to be updated
on the other parties while synchronization. Often different parties use
different ids to identify their entries. So it's required to map the logical
same entries which each others native id. The combination of those different
entries on different parties are called ">Mappings"<. Those ">Mappings"< make it
possible to determine a conflict if mapped entries got changed on different
parties the same time in a different way.
\section{Conflicts}
So called ">Conflicts"< appear if at least two entries of the same mapping got
changed in a different way. No conflict appears if all entries of the mapping
changed the same way. Such conflicts have to be handled by the Synchronization
Framework to avoid data loss. There are several ways to solve such conflicts.
OpenSync provides several different for a proper conflict resolution without
gaining unintended loss of data. Following conflict resolution are supported by
the OpenSync Framework:

\begin{itemize}
\item Solve, means intentionally choosing one of the conflicting entries to
solve the conflict. The chosen one will be multiplied to all parties and will
overwrite the other conflicting changes. This also allows to configure in
advance who is the ">Winning"< Party, who's changes will always used as the 
solving change (">master change"<) for the conflict.
\item Duplicate, (intentionally) will duplicate all changed entries. 
\item Latest, using the latest changed entry of the conflicting entries. This is
only an conflict resolution option if all changes provide within their formats
enough information to determine which got most recently changed.
\item Ignoring, (temporarily) the conflict till the next synchronization.
Conflicting entries will be read and compared again by the OpenSync Framework on
the next synchronization. To avoid inconsistence and data loss. If the
entries/changes are equal on the next synchronization the conflict is solved as
well. (This conflict resolution requires that the protocol of all parties is
able to request single entries, without triggering a "Slow Sync".)
\end{itemize}
\section{Capabilities}
\section{Filter}
OpenSync provides initial code for filtering, but it's not yet usable within
OpenSync 0.40. Looking forward to OpenSync 0.41!
