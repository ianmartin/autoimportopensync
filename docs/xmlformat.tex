\documentclass[a4paper,11pt]{article}

\usepackage{ucs}
\usepackage[utf8]{inputenc}
\usepackage[T1]{fontenc}
\usepackage{acronym}
\usepackage{url}
\usepackage{listings}

\title{Definition of the Capabilities and the XML Format}
\author{Daniel Friedrich \\
	Martin Schunk}
\date{17.05.2006}

%\textless contact \textgreater \\
%$<$contact$<$ \\
%\verb|<xml>|
%\begin{abstract}
%\end{abstract}
%\newpage
%\begin{itemize}
%  \item Diese Liste...
%  \item ...hat anstatt der normalen schwarzen Punkte...
%  \item ...Buchstaben als Marker.
%\end{itemize}

\begin{document}

\maketitle
\tableofcontents

\section{General}
This document will describe the internal xml format of opensync.
The format based on a valid and wellformed xml 1.0 document.
All elements are case sensitve.

Contact -> Object level
"Name" -> Field level
"FirstName" -> Key level
"joe" -> Value level

\subsection{Predefined Attributes}
\subsubsection{TelephoneLocation}
\begin{itemize}
  \item values:
  \subitem Home
  \subitem Work
\end{itemize}

\subsubsection{TelephoneType}
\begin{itemize}
  \item values:
  \subitem Voice
  \subitem Cellular
  \subitem Fax
  \subitem Message
  \subitem Pager
  \subitem Modem
  \subitem Isdn
  \subitem Assistant
  \subitem Callback
  \subitem Company
  \subitem Car
  \subitem Radio
  \subitem Telex
  \subitem TtyYtdd
  \subitem Bbs
  \subitem Video
\end{itemize}

\subsubsection{EmailLocation}
\begin{itemize}
  \item values:
  \subitem Home
  \subitem Work
  \subitem Other
\end{itemize}

\subsubsection{EMailType}
\begin{itemize}
  \item values:
  \subitem Internet
  \subitem IbmMail
  \subitem Aol
  \subitem AppleLink
  \subitem AttMail
  \subitem Cis
  \subitem eWorld
  \subitem MciMail
  \subitem Powershare
  \subitem Prodigy
  \subitem Tlx
  \subitem X400
\end{itemize}

\subsubsection{AddressLocation}
\begin{itemize}
  \item values:
  \subitem Home
  \subitem Work
  \subitem Domestic
  \subitem International
  \subitem Postal
  \subitem Parcel
\end{itemize}

\subsubsection{PhotoFormat}
\begin{itemize}
  \item values:
  \subitem Bmp
  \subitem Gif
  \subitem Cgm
  \subitem Wmf
  \subitem Met
  \subitem Pmb
  \subitem Dib
  \subitem Pict
  \subitem Tiff
  \subitem Ps
  \subitem Pdf
  \subitem Jpeg
  \subitem Mpeg
  \subitem Mpeg2
  \subitem Avi
  \subitem Qtime
\end{itemize}

\subsubsection{PhotoEncode}
\begin{itemize}
  \item values:
  \subitem b
\end{itemize}

\subsubsection{SoundFormat}
\begin{itemize}
  \item values:
  \subitem Wave
  \subitem Pcm
  \subitem Aiff
\end{itemize}

\subsubsection{Messaging}
\begin{itemize}
  \item values:
  \subitem Home
\end{itemize}

\subsubsection{Class}
\begin{itemize}
  \item values:
  \subitem Public
  \subitem Private
  \subitem Confidential
\end{itemize}

\subsubsection{KeyType}
\begin{itemize}
  \item values:
  \subitem X509
  \subitem Pgp
\end{itemize}

\subsubsection{Bool}
\begin{itemize}
  \item values:
  \subitem True
  \subitem False
\end{itemize}

\subsubsection{Preferred}
\begin{itemize}
  \item values:
  \subitem 0
  \subitem 1
\end{itemize}

\section{XML Format}
The root element will be named "xmlformat" and has a version attribute.
It can contain one or more of the following pim types:
\begin{itemize}
  \item Contact
  \item Event
  \item Todo
  \item Note
\end{itemize}
the subelements of a pim type have to be one of the following types: \\
\textbf{container} or \textbf{sequence} \\
If it is from type container than the order of the subelements are equal.
If it is from type sequence than the order of the subelements have to be in this order as they are listed.



\subsection{Contact}

A contact consists of 0 up to $n$ of the following elements.

\subsubsection{Name}
\begin{itemize}
  \item type: sequence
  \item minOccurs: 0
  \item maxOccurs: 1
  \item subnodes:
  \subitem LastName
  \subitem FirstName
  \subitem Additional
  \subitem Prefix
  \subitem Suffix
\end{itemize}

\subsubsection{FormattedName}
\begin{itemize}
  \item type: container
  \item subnodes:
  \subitem Content
\end{itemize}

\subsubsection{Title}
\begin{itemize}
  \item type: container
  \item minOccurs: 0
  \item maxOccurs: 1
  \item subnodes:
  \subitem Content
\end{itemize}

\subsubsection{NickName}
\begin{itemize}
  \item type: container
  \item minOccurs: 0
  \item maxOccurs: 1
  \item subnodes:
  \subitem Content
\end{itemize}

\subsubsection{Birthday}
\begin{itemize}
  \item type: container
  \item minOccurs: 0
  \item maxOccurs: 1
  \item attributes:
  \item subnodes:
  \subitem Content
\end{itemize}

\subsubsection{Anniversary}
\begin{itemize}
  \item type: container
  \item minOccurs: 0
  \item maxOccurs: 1
  \item attributes:
  \item subnodes:
  \subitem Content
\end{itemize}

\subsubsection{Spouse}
\begin{itemize}
  \item type: container
  \item minOccurs: 0
  \item maxOccurs: 1
  \item attributes:
  \item subnodes:
  \subitem Content
\end{itemize}

\subsubsection{Manager}
\begin{itemize}
  \item type: container
  \item minOccurs: 0
  \item maxOccurs: 1
  \item subnodes:
  \subitem Content
\end{itemize}

\subsubsection{Organization}
\begin{itemize}
  \item type: sequenz
  \item minOccurs: 0
  \item maxOccurs: 1
  \item attributes:
  \item subnodes:
  \subitem Name
  \subitem Department
  \subitem Unit
\end{itemize}

\subsubitem{Assistant}
\begin{itemize}
  \item type: sequenz
  \item minOccurs: 0
  \item maxOccurs: 1
  \item attributes:
  \item subnodes:
  \subitem Content
\end{itemize}

\subsubsection{Role}
\begin{itemize}
  \item type: container
  \item minOccurs: 0
  \item maxOccurs: 1
  \item attributes:
  \item subnodes:
  \subitem Content
\end{itemize}

\subsubsection{Address}
\begin{itemize}
  \item type: sequence
  \item minOccurs: 0
  \item maxOccurs: n
  \item attributes: AddressLocation, Preferred
  \item subnodes:
  \subitem Postofficebox
  \subitem Extendedaddress
  \subitem Street
  \subitem Locality (e.g., city)
  \subitem Region (e.g., state or province)
  \subitem PostalCode
  \subitem Country
\end{itemize}

\subsubsection{AddressLabel}
\begin{itemize}
  \item type: container
  \item minOccurs: 0
  \item maxOccurs: 1
  \item attributes: AddressLocation
  \item subnodes:
  \subitem Content
\end{itemize}

\subsubsection{Location}
\begin{itemize}
  \item type: sequenz
  \item minOccurs: 0
  \item maxOccurs: 1
  \item attributes:
  \item subnodes:
  \subitem Latitude
  \subitem Longitude
\end{itemize}

\subsubsection{Telephone}
\begin{itemize}
  \item type: container
  \item attributes: Preferred, TelphoneType, TelephoneLocation
  \item subnodes:
  \subitem Content
\end{itemize}

\subsubsection{EMail}
\begin{itemize}
  \item type: container
  \item attributes: EmailLocation, EmailType, Preferred
  \item minOccurs: 0
  \item maxOccurs: 1
  \item subnodes:
  \subitem Content
\end{itemize}

\subsubsection{Photo}
\begin{itemize}
  \item type: container
  \item minOccurs: 0
  \item maxOccurs: 1
  \item attributes: PhotoFormat, PhotoEncode
  \item subnodes:
  \subitem Content
\end{itemize}

\subsubsection{PhotoUrl}
\begin{itemize}
  \item type: container
  \item minOccurs: 0
  \item maxOccurs: 1
  \item attributes:
  \item subnodes:
  \subitem Content
\end{itemize}

\subsubsection{Url}
\begin{itemize}
  \item type: container
  \item minOccurs: 0
  \item maxOccurs: 1
  \item attributes:
  \item subnodes:
  \subitem Content
\end{itemize}

\subsubsection{Catagories}
\begin{itemize}
  \item type: sequenz
  \item minOccurs: 0
  \item maxOccurs: 1
  \item attributes:
  \item subnodes:
  \subitem Catagory
\end{itemize}

\subsubsection{Note}
\begin{itemize}
  \item type: container
  \item minOccurs: 0
  \item maxOccurs: 1
  \item attributes:
  \item subnodes:
  \subitem Content
\end{itemize}

\subsubsection{Icq}
\begin{itemize}
  \item type: container
  \item minOccurs: 0
  \item maxOccurs: n
  \item attributes: Messaging
  \item subnodes:
  \subitem Content
\end{itemize}

\subsubsection{Msn}
\begin{itemize}
  \item type: container
  \item minOccurs: 0
  \item maxOccurs: n
  \item attributes: Messaging
  \item subnodes:
  \subitem Content
\end{itemize}

\subsubsection{Aim}
\begin{itemize}
  \item type: container
  \item minOccurs: 0
  \item maxOccurs: n
  \item attributes: Messaging
  \item subnodes:
  \subitem Content
\end{itemize}

\subsubsection{GroupWise}
\begin{itemize}
  \item type: container
  \item minOccurs: 0
  \item maxOccurs: n
  \item attributes: Messaging
  \item subnodes:
  \subitem Content
\end{itemize}

\subsubsection{Yahoo}
\begin{itemize}
  \item type: container
  \item minOccurs: 0
  \item maxOccurs: n
  \item attributes: Messaging
  \item subnodes:
  \subitem Content
\end{itemize}

\subsubsection{Yabber}
\begin{itemize}
  \item type: container
  \item minOccurs: 0
  \item maxOccurs: n
  \item attributes: Messaging
  \item subnodes:
  \subitem Content
\end{itemize}

\subsubsection{Class}
\begin{itemize}
  \item type: container
  \item minOccurs: 0
  \item maxOccurs: 1
  \item attributes:
  \item subnodes:
  \subitem Content
\end{itemize}

\subsubsection{WantsHTML}
\begin{itemize}
  \item type: container
  \item minOccurs: 0
  \item maxOccurs: 1
  \item attributes: Bool 
\end{itemize}

\subsubsection{VideoUrl}
\begin{itemize}
  \item type: container
  \item minOccurs: 0
  \item maxOccurs: 1
  \item attributes:
  \item subnodes:
  \subitem Content
\end{itemize}

\subsubsection{BlogUrl}
\begin{itemize}
  \item type: container
  \item minOccurs: 0
  \item maxOccurs: 1
  \item attributes:
  \item subnodes:
  \subitem Content
\end{itemize}

\subsubsection{CalUri}
\begin{itemize}
  \item type: container
  \item minOccurs: 0
  \item maxOccurs: 1
  \item attributes:
  \item subnodes:
  \subitem Content
\end{itemize}

\subsubsection{FbUrl}
\begin{itemize}
  \item type: container
  \item minOccurs: 0
  \item maxOccurs: 1
  \item attributes:
  \item subnodes:
  \subitem Content
\end{itemize}

\subsubsection{Sound}
\begin{itemize}
  \item type: container
  \item minOccurs: 0
  \item maxOccurs: 1
  \item attributes: SoundFormat
  \item subnodes:
  \subitem Content
\end{itemize}

\subsubsection{Version}
\begin{itemize}
  \item type: container
  \item minOccurs: 0
  \item maxOccurs: 1
  \item attributes:
  \item subnodes:
  \subitem Content
\end{itemize}

\subsubsection{Uid}
\begin{itemize}
  \item type: container
  \item minOccurs: 0
  \item maxOccurs: 1
  \item attributes:
  \item subnodes:
  \subitem Content
\end{itemize}

\subsubsection{Rev}
\begin{itemize}
  \item type: container
  \item minOccurs: 0
  \item maxOccurs: 1
  \item attributes:
  \item subnodes:
  \subitem Content
\end{itemize}

\subsubsection{Key}
\begin{itemize}
  \item type: container
  \item minOccurs: 0
  \item maxOccurs: 1
  \item attributes: KeyType
  \item subnodes:
  \subitem Content
\end{itemize}

\subsubsection{UserDefined}
\begin{itemize}
  \item type: container
  \item minOccurs: 0
  \item maxOccurs: 1
  \item attributes:
  \item subnodes:
  \subitem Content
\end{itemize}

\begin{lstlisting}
<?xml version="1.0" encoding="UTF-8" standalone="yes"?>
<xmlformat>
  <contact version="1">
    <name>
      <FirstName>John</FirstName>
      <LastName>Doe</LastName>
    </Name>
    <Telephone type="Cellular">
      <Content>123</Content>
    </Telephone>
  </contact>
</xmlformat>
\end{lstlisting}

\subsection{Todo}
\subsection{Event}
\subsection{Note}

\section{Capabilities}

The capabilities will be used by the merging problem. The problem is that no
existing format like vcard or so have defined some id for each field of a record.
So there is no way to find out which field has changed or not.
The minimal solution is to add or don't to add fields generaly.
The xmlformat should be easily extensible in the future by adding some id attribut.
If a subelement of a pim type have a occurs of max. 1 than it is allowed to
specify the subelements of this element in the capabilities.

It can contain one or more of the following object type elements:
\begin{itemize}
  \item Contact
  \item Event
  \item Todo
  \item Note
\end{itemize}



\begin{lstlisting}
<?xml version="1.0" encoding="UTF-8" standalone="yes"?>
<capabilities>
  <contact>
    <name>
      <firstname />
      <lastname />
    </name>
    <telephone />
  </contact>
</capabilities>
\end{lstlisting}

%\appendix

%\begin{acronym}
%\acro{XML}  {Extensible Markup Language}
%\end{acronym}

%\begin{thebibliography}{99}
%\bibitem{W3C} \url{http://www.w3.org/}, Definition of XML
%\end{thebibliography}

\end{document}
